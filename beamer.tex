\documentclass{beamer}

\usetheme[progressbar=frametitle]{metropolis}
\setbeamertemplate{frame numbering}[fraction]
\useoutertheme{metropolis}
\useinnertheme{metropolis}
\usefonttheme{metropolis}
\usecolortheme{spruce}
\setbeamercolor{background canvas}{bg=white}

\usepackage{xeCJK}
\setCJKmainfont{Source Han Sans CN}
\usefonttheme{professionalfonts}

\usepackage{multicol}

\setbeamercovered{transparent=15}

\begin{document}
\metroset{block=fill}

\title{Functions, Limits, Derivatives}
% \subtitle{Sub Title Here}
\author{}
\institute{\large \textbf{Arch Linux:} \\[3pt] defines simplicity as without unnecessary additions or modifications.}
\date{}

\begin{frame}
  \titlepage
\end{frame}

\begin{frame}[t]{Functions}\vspace{4pt}
  \begin{block}{Definition of a function}
    \vspace{4pt}
    A \textbf{function} $f$ is a rule that assigns to each element $x$ in a set $D$ exactly one element, called $f(x)$, in a set $E$.
    \vspace{4pt}
  \end{block}

  Set $D$ is called the
  \only<1>{\line(1,0){50}}
  \only<2>{\textcolor{magenta}{domain}}
  \, of the function.\\[10pt]
  Set $E$ is called the
  \only<1>{\line(1,0){50}}
  \only<2>{\textcolor{magenta}{range}}
  \, of the function.
\end{frame}

\begin{frame}[t]{Flash Card} \vspace{4pt}
  $\sqrt{x^{2}}=$\\ \vspace{10pt}
  \begin{enumerate}[(A)]
  \item $x$
  \item $-x$
  \item $|x|$
  \item undefined
  \end{enumerate}

\end{frame}

\begin{frame}[t]{Flash Card2} \vspace{4pt}

  \begin{columns}[onlytextwidth]
    \column{0.5\textwidth}
    $\sqrt{x^{2}}=$\\ \vspace{10pt}
    \begin{enumerate}[(A)]
    \item $x$
    \item $-x$
    \item $|x|$
    \item undefined
    \end{enumerate}

    \column{0.5\textwidth}
    \only<3> {
      $\sqrt{x^{2}}=
      \begin{cases}
        -x, & x<0\\
        x, & x \geq 0
      \end{cases}$\\ [4pt]
    }
    \only<2-> {
      \includegraphics[scale=0.25]{./static/profile}
    }
  \end{columns}

\end{frame}

\begin{frame}[t]{multi columns} \vspace{4pt}

  \begin{enumerate}
    \begin{multicols}{3}
    \item a
    \item a
    \item a
    \item a
    \item a
      \onslide<2->{
      \item a
      \item a
      \item a
      \item a
      \item a
      \item a
      }
      \onslide<3->{
      \item a
      \item a
      \item a
      \item a
      \item a
      \item a
      }
    \end{multicols}
  \end{enumerate}

\end{frame}

\begin{frame}[standout]
  \flushleft
  Thanks!
\end{frame}

\end{document}