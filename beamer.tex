\documentclass{beamer}

\usetheme[progressbar=frametitle]{metropolis}
\setbeamertemplate{frame numbering}[fraction]
\useoutertheme{metropolis}
\useinnertheme{metropolis}
\usefonttheme{metropolis}
\usecolortheme{spruce}
\setbeamercolor{background canvas}{bg=white}

\usepackage{xeCJK}
\setCJKmainfont{Source Han Sans CN}
\usefonttheme{professionalfonts}

\begin{document}
\metroset{block=fill}

\title{Functions, Limits, Derivatives}
% \subtitle{Sub Title Here}
\author{}
\institute{\large \textbf{Arch Linux:} \\[3pt] defines simplicity as without unnecessary additions or modifications.}
\date{}

\begin{frame}
  \titlepage
\end{frame}

\begin{frame}[t]{Functions}\vspace{4pt}
  \begin{block}{Definition of a function}
    \vspace{4pt}
    A \textbf{function} $f$ is a rule that assigns to each element $x$ in a set $D$ exactly one element, called $f(x)$, in a set $E$.
    \vspace{4pt}
  \end{block}

  Set $D$ is called the
  \only<1>{\line(1,0){50}}
  \only<2>{\textcolor{magenta}{domain}}
  \, of the function.\\[10pt]
  Set $E$ is called the
  \only<1>{\line(1,0){50}}
  \only<2>{\textcolor{magenta}{range}}
  \, of the function.
\end{frame}


\end{document}